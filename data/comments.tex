% !TeX root = ../thuthesis-example.tex

\begin{comments}
% \begin{comments}[name = {指导小组学术评语}]
% \begin{comments}[name = {Comments from Thesis Supervisor}]
% \begin{comments}[name = {Comments from Thesis Supervision Committee}]

论文研究时间序列对称模式挖掘,用于轨迹分析、异常识别和序列预测等。选题具有实际应用价值。 
  
论文主要工作包括:  
 
1. 针对单一对称模式,提出了一种对称性度量,以及自适应的对称度阈值确定方法,并设计了最优对称模式求解算法; 
 
2. 针对分段对称模式,提出了基于滑动窗口的时间序列分段处理方法,解决对称子模式重叠问题; 
 
3. 对上述方法进行了实验评估,并在工业物联网时序数据库Apache IoTDB中进行了系统实现。 
 
论文结构清晰,目标明确,通过以上工作表明,李盼盼同学掌握了软件工程的基础理论和对应的专业知识,具备了解决工程问题的方法与手段,具备了独立展开工程技术工作的能力,达到了工程硕士的学术水平,同意组织论文答辩。

\end{comments}
