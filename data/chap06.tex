% !TeX root = ../thuthesis-example.tex

\chapter{系统实现}
本章将将介绍全局对称模式挖掘算法和分段对称模式挖掘算法的系统实现和应用。
首先对集成这两个算法的IoTDB-Quality工具进行介绍,在介绍算法的相关配置和
应用。

\section{IoTDB-Quality介绍}
IoTDB-Quality基于IoTDB用户自定义函数(UDF),
实现了一系列关于数据质量和分析的函数,
包括数据画像、数据质量评估与修复、数据匹配和模式发现等,
有效满足了工业领域对数据挖掘的需求。
当对一个长时间序列进行预测和异常检测时,
如果能提前识别时间序列中蕴含的模式,将能够对时间序列未来的发展变化
具有一个清晰明确的预期,为用户提供预测或修复的有效参考。
工业物联网中的数据往往是来自实际应用场景的工况信息,对称模式在其中
广泛存在。利用对称模式挖掘算法,不仅可以发现蕴含在时间序列中的模式信息,
,而且可以通过统计对时间序列的分段特征进行量化,
为IoTDB-Quality的功能进行了扩展。

\section{全局对称模式挖掘算法实现}


\section{分段对称模式挖掘算法实现}