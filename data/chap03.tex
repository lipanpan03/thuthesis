% !TeX root = ../thuthesis-example.tex

\chapter{对称模式挖掘算法}

本章将进一步对所提出的对称模式挖掘方法进行具体阐述。
整体来看,时间序列对称模式挖掘分为两大类,
一类是时间序列整体组成一个全局的对称模式,
另一类是对时间序列进行分段处理,由对称子序列组成的对称模式集合。
对于第一类对称模式挖掘,需要一个全局对称性度量算法和
基于时间序列数据特征的对称度阈值算法以过滤对称模式。
而若需要计算分段时间序列的对称性,
则需要在基于分段长度约束的前提下,
首先对时间序列进行分段处理,
之后通过分段对称性度量算法计算时间子序列的对称度,
然后使用由数据特征和对称度分布特征共同确定的对称度阈值
分类得到对称子序列,
最后根据挖掘数量最多不重叠对称子序列的贪心策略挖掘出对称模式。

在对时间序列的对称度进行度量之前,需要执行一步非常重要的操作,数据归一化。
这是一个在全局对称模式挖掘和分段对称模式挖掘之前都要进行的操作。由于不同来源的时间序列
数据范围往往差距很大,图~\ref{fig:data_range}展示了运输车运输过程中的经度工况时间序列和
有规律心跳过程中力感测电阻器(FSR)信号的变化时间序列,运输车经度的在
(106.89,106.9)范围内,而FSR信号在(-4000,-1000)范围内。
除了范围之外,很明显,两个数据来源的数据跨度差别也很大。
为使用统一的算法框架进行对称模式挖掘,需要对数据进行标准化处理。
对时间序列$X=\left(\left(t_{1}, x_{1}\right),\left(t_{2}, x_{2}\right), \ldots,\left(t_{n}, x_{n}\right)\right)$
的每个数据点,首先根据公式~\ref{eq:mu}计算数据点的平均值,
再根据公式~\ref{eq:sigma}计算数据点的标准差,
最后按照公式~\ref{eq:standard}对时间序列X中所有的数据点进行标准化。
使用z-score方法进行标准化,不仅在无量纲化过程中利用了所有的数据信息,
还消除了各变量在变异程度上的差异,方便提出统一的对称度阈值确定算法。
\begin{equation}
  \mu=\frac{\sum_{i=1}^{i=n} x_{i}}{n}
  \label{eq:mu}
\end{equation}
\begin{equation}
  \sigma=\sqrt[2]{\frac{\sum_{i=1}^{i=n}\left(x_{i}-\mu\right)^{2}}{n}}
  \label{eq:sigma}
\end{equation}
\begin{equation}
  y_{i}=\frac{x_{i}-\mu}{\sigma}
  \label{eq:standard}
\end{equation}
\begin{figure}
  \centering
  \subcaptionbox{挖掘机经度\label{fig:data_range-a}}
  {\includegraphics[width=0.43\linewidth]{truck_lo.png}}
  \subcaptionbox{心率FSR信号\label{fig:data_range-b}}
  {\includegraphics[width=0.43\linewidth]{heart_fsr.png}}
  \caption{多源时间序列数据分布范围}
  \label{fig:data_range}
\end{figure}

\section{全局对称模式挖掘算法}
时间序列的全局对称性指的是时间序列的所有点关于某个对称中心前后互为镜像。
全局对称模式挖掘就是要通过计算时间序列的对称度,判断时间序列是否具有对称性。
从数学意义上来说,全局对称模式挖掘算法是分段对称模式挖掘算法的子问题,
全局算法不需要以分段长度$w$作为约束,直接度量全局的对称性,并以全局对称性
在对称度阈值范围内的时间序列整体做为对称模式。
因此,全局对称模式挖掘算法由两部分构成,分别是全局时间序列对称性度量算法和
全局对称度阈值确定算法,本节将分别介绍这两种算法。
\subsection{全局时间序列对称性度量算法}

对于全局时间序列而言,使用原始时间序列和其反转时间序列的相似性来度量对称性,
是一种可行的时间序列对称性度量方法。此方法的好处是可以避免确定对称中心
的问题,但是,这种方法实际上是将时间序列对称中心两侧的子序列进行了
两次匹配,不仅降低了对称性度量算法的时间效率,还增大了对称性的度量结果。
图~\ref{fig:beijing_temp}展示了北京市自1981年至2010年月平均气温的变化,显然,
气温时间序列具有对称性。但是,经计算,原始和反转时间序列的欧式距离为
278.04,而使用本文所提出的对称性度量方法,气温时间序列的对称度仅为
7.6,更加符合真实的时间序列对称性结果。因此,本文定义了一种新的基于动态规整算法和区间动态规划思想的对称性
度量方式,既保持了动态时间规整算法准确率高、鲁棒性强的特点,
又优化了对称度的结果和计算效率。

首先从全局角度考虑对称时间序列的
匹配过程,由于不存在明确的对称中心,只能通过点和点之间的直接匹配度量
序列的对称性。由于时间序列的采集频率和位置不同,对于每个单点而言,
并不确定最佳匹配点的位置。尽管如此,如果一个时间序列是前后对称的,
那么首尾点一定是匹配的。图~\ref{fig:frontend_match}展示了
对称时间序列匹配过程的候选点,对于时间序列
$X=\left(\left(t_1,x_1 \right),\left(t_2,x_2\right),\dots,
  \left(t_n,x_n \right)\right)$而言,
第一个点$x_1$和最后一个点$x_n$是必然匹配的,但由于时间序列在
同一位置的持续时间不同,其他点如$x_2$和$x_{n-1}$等的匹配点却不是
唯一的,$x_2$可以一一对应地与$x_{n-1}$进行匹配,也可以通过扭曲时间
和$x_n$进行匹配,甚至可以与$x_{n-2},x_{n-3},\dots$等进行匹配,
而制约$x_2$与$x_k$匹配的条件不仅仅是这两点之间的距离,而是
时间序列$\left(\left(t_2,x_2 \right),\dots,\left(t_k,x_k \right)\right)$
的对称度。
\begin{figure}
  \centering
  \includegraphics[width=0.86\linewidth]{beijing_temp.png}
  \caption{北京市月平均气温变化}
  \label{fig:beijing_temp}
\end{figure}

\begin{figure}
  \centering
  \includegraphics[width=0.76\linewidth]{frontend_match.png}
  \caption{对称时间序列首尾匹配候选点}
  \label{fig:frontend_match}
\end{figure}

接下来,本节将对时间序列的对称度算法进行标准的形式化推导。给定一条时间序列$X=\left(\left(t_1,x_1 \right),\left(t_2,x_2\right),\dots,
  \left(t_n,x_n \right)\right)$,以任意两点间的距离确立$n \times m$的
距离矩阵$D_{n \times m}$,矩阵中的每个元素由公式~\ref{eq:distance_matrix}计算而来
\begin{equation}
  D\left(i, j\right)=\left\|x_{i}-x_{j}\right\|_{w}
  \label{eq:distance_matrix}
\end{equation}

$D\left(i, j\right)$为点$p_i$与$p_j$间的距离,$i,j=1,2,\dots,n$。
显然,点$p_i$和$p_j$的匹配与$p_j$和$p_i$的匹配距离是相同的,
只是匹配顺序不同。为避免重复计算,本文只保留前点和后点的匹配,
则距离矩阵$D_{n \times m}$只需要计算反对角线以上的斜三角矩阵即可。
为了计算$X$的对称度,需要找到一条最优的匹配路径
$R_{best}=\left(r_1,r_2,\dots,r_k \right),\left⌊n/2\right⌋ \leq
  k < n$,使得$X$的累积匹配距离值达到最小, $r_k$表示该匹配路径元素在
距离矩阵中的位置,即$r_k=\left(i,j\right)_k$表示$p_i$与$p_j$之间的匹配关系,
可知$D\left(r_k \right)=D\left(i,j\right)_k$一般存在着多条匹配路径,
有效的弯曲路径$R$必须符合以下3个条件:
\begin{enumerate}
  \item 边界性:$r_{1} \in\{(i, i),(i, i+1) \mid 1 \leq i \leq n\}, \quad r_{k}=(1, n)$
  \item 单调性:给定$r_{k}=(i, j)$和$r_{k+1}=\left(i^{\prime}, j^{\prime}\right), \quad i^{\prime} \leq i, j^{\prime} \geq j$
  \item 连续性:给定$r_{k}=(i, j)$和$r_{k+1}=\left(i^{\prime}, j^{\prime}\right), \quad i^{\prime} \geq i-1, j^{\prime} \leq j+1$
\end{enumerate}

边界性是确保$R$的起点$r_1$在相似度矩阵的反对角线或者相邻斜线上,
而终点$r_k$在矩阵的右上角$\left(1,n\right)$。
单调性和连续性是为了保证匹配路径的下一个点在当前点的上方、右上方或右方。
在所有有效的路径中, 找到唯一且最优的路径使得累积匹配距离和达到最小,
公式~\ref{eq:best_route}即为优化目标:
\begin{equation}
  D(X)=\min \left\{\sum_{k=1}^{K} D\left(r_{k}\right)\right\}
  \label{eq:best_route}
\end{equation}

然而,从$r_1$到$r_k$的有效路径却是指数级的,采用暴力求解的方法不现实。
图~\ref{fig:symmetric_matrix}展示了一个时间序列在对称度矩阵上的最优
匹配路径。观察发现,路径R中的点$r_i$对于时间序列而言是由内向外匹配的。
如果$r_{k+1}$与$r_k=\left(i,j\right)$属于同一个匹配路径中的相邻点,
则$r_{k+1}$只有$(i-1, j),(i-1, j+1),(i, j+1)$这三种可能,
即$r_{k+1}$所表示的匹配范围正好包含了$r_k$的匹配范围,这种匹配顺序
符合了动态规划推导思想的无后效性。进一步,如果约定$D P(i, j)$表示时间序列
$X$的子序列$S=\left(p_{i}, p_{i+1}, \dots, p_{j}\right)$的对称度,
则该状态可以由$\{D P(i+1, j), D P(i, j-1), D P(i+1, j-1)\}$中
最小的一个推导而来,这种小区间和大区间的状态具有严格单调性的模型非常
符合区间动态规划算法思想。因此,为了求解式~\ref{eq:best_route},
利用区间动态规划方法构造一个代价矩阵$DP$, 其中每个元素通过式~\ref{eq:dp_item}得到:
\begin{equation}
  D P(i, j)=D(i, j)+\min \left\{\begin{array}{c}
    D P(i+1, j) \\
    D P(i, j-1) \\
    D P(i+1, j-1)
  \end{array}\right.
  \label{eq:dp_item}
\end{equation}

其中:$1 \leq i \leq j \leq n, D P(i, i)=0, D P(i, i+1)=D(i, i+1)$。
式~\ref{eq:dp_item}表示当前区间的匹配累积距离等于当前点的距离值加上紧邻
3个区间匹配累积距离的最小值,$D P(1, n)$便是时间序列X的最小匹配累积
代价,即对称度。得到对称度后, 为了得到最优匹配路径,
再反向以$r_k$为起点寻找匹配路径。直到$i=j$或者$i=j-1$时,
寻找过程结束,最终得到完整的匹配路径。通过累积计算匹配路径中每对匹配点之间的距离,
可以得到匹配点的距离之和,最终计算得到的$DP(1,n)$即可视为时间序列$X$的对称度。
\begin{figure}
  \centering
  \includegraphics[width=0.86\linewidth]{symmetric_matrix.png}
  \caption{相似度矩阵最优匹配路径示意图}
  \label{fig:symmetric_matrix}
\end{figure}

\renewcommand{\algorithmicrequire}{\textbf{输入:}\unskip}
\renewcommand{\algorithmicensure}{\textbf{输出:}\unskip}

\begin{algorithm}
  \caption{全局时间序列对称性度量算法$calculate\_global\_symmetry$}
  \label{alg:global_symmetry}
  \small
  \begin{algorithmic}
    \REQUIRE 时间序列$X=\left(p_{1}, p_{2}, \dots, p_{n}\right)$
    \ENSURE 对称度$d$

    \STATE $n \leftarrow \left|X\right|$
    \STATE $i \leftarrow 1$
    \WHILE{$i \leq n$}
    \STATE $dp_{i,i} \leftarrow inf$
    \ENDWHILE

    \STATE $i \leftarrow 1$
    \WHILE{$i < n$}
    \STATE $dp_{i,i+1} \leftarrow D\left(p_{i}, p_{i+1}\right)$
    \ENDWHILE

    \STATE $len \leftarrow 3$
    \WHILE{$len \leq n$}
    \STATE $i \leftarrow len$
    \WHILE{$i \leq n-len+1$}
    \STATE $dp_{i,i+len-1} = D\left(p_{i}, p_{i+1}\right)+\min \left(dp_{i,i+len-2},dp_{i+1,i+len-1},dp_{i+1,i+len-2}\right)$
    \ENDWHILE
    \ENDWHILE
    \RETURN $dp_{1,n}$
  \end{algorithmic}
\end{algorithm}

算法~\ref{alg:global_symmetry}给出了
全局时间序列对称性度量算法的计算流程。
第1-9行初始化计算长度为1和2的时间子序列对称度,
第10-17行根据式~\ref{eq:dp_item}所示的区间动态规划算法
度量得到时间序列的全局对称度。

总结来说,本方法利用区间动态规划的算法思想,提出了一种兼顾距离相近和形状相似的时间序列对称性
度量方法,既避免了时间序列对称中心不固定带来的匹配问题,也通过异步匹配
提高了度量的准确性和健壮性,同时,因减少了重复匹配,也提高了本算法的
时间效率。尽管其渐进时间复杂度仍然为$O\left(w^{2}\right)$,
但相比于基于原始和反转时间序列的DTW距离的相似度度量算法,
减少了$50 \%$的重复匹配,并且在3.2节所述的分段时间序列对称性度量算法中
更加提升了对称子序列的计算效率。

\subsection{全局对称度阈值确定算法}
3.1.1节讲述了时间序列的全局对称性度量算法,在得到全局时间序列的对称度之后,
需要确定对称度阈值,才能判断时间序列是否具有对称性,进而挖掘出全局对称模式。
对称度阈值实际上是对时间序列进行分类,对称度在阈值范围内的
划分为对称序列,否则划分为非对称序列。
因此,对称度阈值必然和时间序列本身的差异程度
密切相关。
如果时间序列的变化比较平缓,为了将对称模式挖掘出来,
对称度阈值就需要设定的较小。
如果时间序列变化特别剧烈,相应的在进行对称时间序列匹配的过程中,
匹配点的差异可能就特别大,因此,对称度阈值也需要
设定的较大。然而,影响对称度阈值高低的关键因素不是标准差
代表的整体时间序列的差异程度,而是匹配点之间的差异程度。
图~\ref{fig:symmetry_diff}展示了标准化后正弦曲线和河北某县气温的变化时间序列,
两者都是对称时间序列。然而,尽管两者的标准差都是1,
但因为正弦时间序列完美匹配而气温时间序列匹配存在误差,
所以前者的对称度为0,而后者的对称度为0.7。
\begin{figure}
  \centering
  \subcaptionbox{标准化正弦时间序列\label{fig:symmetry_diff_sin}}
  {\includegraphics[width=0.43\linewidth]{symmetry_diff_sin.png}}
  \subcaptionbox{标准化气温时间序列\label{fig:symmetry_diff_temp}}
  {\includegraphics[width=0.43\linewidth]{symmetry_diff_temp.png}}
  \caption{多源对称时间序列对称度差异}
  \label{fig:symmetry_diff}
\end{figure}

因此,要根据时间序列点匹配的情况确定对称度阈值。然而,
根据时间序列匹配点之间的距离确定对称阈值具有极大的随机性。
对称度阈值的选取将极大地受对称性度量算法的影响。
因为对称性度量算法将决定时间序列点的匹配位置。
这样得到的对称度阈值不具有统一性,算法也不具备可迁移性。因此,
对称度阈值的确定还是需要立足时间序列本身的数据特征。
本文使用在同一个时间序列中前后相邻点的
差距作为确定对称度阈值的参考。因为对称时间序列要求前后点相匹配,
符合要求的匹配应该满足匹配点对的距离尽可能小,进而也尽量不要超过匹配点
的近邻点。式~\ref{eq:threshold1}表示时间序列近邻点距离的平均值,
可作为全局时间序列对称模式挖掘算法的对称度阈值。
对称度阈值需要统一量纲,由于不同来源时间序列对称模式的长度不一致,
因而经全局时间序列对称性度量算法计算出来的对称度
也要求其与对称模式长度的比值。算法~\ref{alg:threshold1}详述了
全局对称度阈值的计算方式。
\begin{equation}
  \theta_{1}=\frac{\sum_{i=2}^{n}\left|x_{i}-x_{i-1}\right|}{n-1}
  \label{eq:threshold1}
\end{equation}

\renewcommand{\algorithmicrequire}{\textbf{输入:}\unskip}
\renewcommand{\algorithmicensure}{\textbf{输出:}\unskip}
\begin{algorithm}
  \caption{全局对称度阈值确定算法$calculate\_global\_threshold$}
  \label{alg:threshold1}
  \small
  \begin{algorithmic}
    \REQUIRE 时间序列$X=\left(p_{1}, p_{2}, \dots, p_{n}\right)$
    \ENSURE 全局对称度阈值$\theta$

    \STATE $i \leftarrow 1$
    \WHILE{$i < \left|X\right|$}
    \STATE minus $\leftarrow$ minus $+D\left(p_{i}, p_{i+1}\right)$
    \STATE $i \leftarrow i+1$
    \ENDWHILE
    \STATE $\theta \leftarrow {\text { minus }} / (n-1)$
    \RETURN $\theta$
  \end{algorithmic}
\end{algorithm}

\section{分段对称模式挖掘算法}

3.1节介绍了全局对称模式挖掘算法,
利用全局时间序列对称性度量算法得到对称度,
再根据全局对称度阈值确定算法得到对称度阈值,
从而挖掘出全局对称模式。然而,真实的对称模式往往是聚合在一条
长时间序列之中的,其中有缺失点、异常点、噪声点等种种干扰,
而且多个对称子序列可能相互重叠在一起,
使得分段对称模式挖掘和结果判断更加困难。
因此,本节设计了一种从长时间序列中挖掘出分段对称子模式的算法。

\subsection{分段时间序列对称性度量算法}
要对时序数据进行对称子序列的挖掘,首先要做的就是把时间序列划分成
一段一段的子序列,然后对子序列进行对称度度量。
时间序列断点检测算法是一种经典的时间子序列划分方法。
图~\ref{fig:break_detection}展示了时间序列断点检测算法的计算流程。为了检测时间序列中的关键点,
该算法采用直线对整体的时间序列数据进行拟合,如果时序数据的趋势发生了变化,
则用多条直线拟合整条时序数据。该算法使用单变量线性回归模型拟合每段时间
序列,每处理一个新数据点就重新计算多段拟合误差,利用动态规划算法全局
最大化分段效果。时间序列断点检测算法的好处是可以根据时间序列的趋势进行
关键点的判断,但是,工业时间序列中的对称模式多种多样,除去首尾点之外,
模式内部可能也存在关键点,只识别出关键点无法成功分割出对称模式。
此外,时间序列断点检测算法的复杂度高达$O\left(n^{3}\right)$,
甚至超出了对称模式挖掘算法的复杂度,在实际工业场景中不具备应用价值。
\begin{figure}
  \centering
  \includegraphics[width=0.86\linewidth]{break_detection.png}
  \caption{时间序列断点检测与直线拟合图}
  \label{fig:break_detection}
\end{figure}

基于此,本文选择基于滑动窗口的时间序列分段算法。滑动窗口源于网络流量控制
技术,在时间序列分析领域可以用于在特定窗口大小的子序列上执行对称度计算的
操作。通过维护一个窗口,不断向前滑动,从而划分出全部的时间序列分段。
图~\ref{fig:sliding_window}展示了使用滑动窗口将时间序列中每个长度为$w$的子序列划分出来的过程。

\begin{figure}
  \centering
  \includegraphics[width=0.86\linewidth]{sliding_window.png}
  \caption{滑动窗口时间序列分段示例}
  \label{fig:sliding_window}
\end{figure}

在滑动窗口模型中,可以直接使用3.1节所述的全局对称性度量算法计算每段时间子序列
的对称性,由于单个时间序列对称性计算的时间复杂度为$O(w^2 )$,所以全部
分段时间序列对称性度量算法的时间复杂度为$O\left(|X| \times w^{2}\right)$。
然而,利用滑动窗口和区间动态规划算法的特点可以将复杂度降低一个阶数。
考虑时间序列$X$两段长度为$w$的连续子序列$S_{i}=\left(\left(t_{i}, x_{i}\right),\left(t_{i+1}, x_{i+1}\right), \dots,\left(t_{i+w-1}, x_{i+w-1}\right)\right)$
和$S_{i+1}=\left(\left(t_{i+1}, x_{i+1}\right),\left(t_{i+2}, x_{i+2}\right), \dots,\left(t_{i+w}, x_{i+w}\right)\right)$的对称度度量,
根据式~\ref{eq:dp_item}的状态方程,时间子序列$S_i$的对称度
$D P(i, i+w-1)$由$DP(i,i+w-2)$,$DP(i+1,i+w-1)$和$DP(i+1,i+w-2)$
的最小值推导而来,而子序列$S_{i+1}$的对称度$D P(i+1, i+w)$由
$D P(i+1, i+w-1)$,$DP(i+2,i+w)$和$DP(i+2,i+w-1)$的最小值推导
而来,这两个对称度的计算都使用到了状态$DP(i+1,i+w-1)$,如果分别计算
将会产生大量的重复计算。然而,考虑到动态规划方程的无后效性,如果先计算
出时间序列$X$中所有长度为$w-1$和$w-2$的子序列对称度并保存下来,
则$S_i$和$S_{i+1}$的状态可以直接计算得到。
图~\ref{fig:fregment}展示了分段时间序列对称度推导过程,
假设时间序列$X$的长度为9,子序列长度即滑动窗口长度为5,采用窗口长度由小到大的
自底向上的推导顺序,每段子序列对称度状态都可以通过$O(1)$的复杂度计算得到,最终所有时间子序列
对称度度量的复杂度由状态个数决定,最底层长度为1的状态有$|X|$个,最顶层
长度为$w$的状态有$|X|-w+1$个,根据等差数列求和共有
$2 \times|X| \times w-w^{2}+w$个状态。
因此,分段时间序列对称度度量算法的渐进时间复杂度为$O(|X| \times w)$,
比通过原始和反转时间序列DTW距离度量对称度的$O\left(|X| \times w^{2}\right)$
效率高一个阶数。
\begin{figure}
  \centering
  \includegraphics[width=0.86\linewidth]{fregment.png}
  \caption{分段时间序列推导流程}
  \label{fig:fregment}
\end{figure}

\subsection{分段对称度阈值确定算法}
3.2.1节讲述了分段时间序列的对称性度量算法,根据3.2.1节的算法,
给定时间序列X和子序列的长度约束$w$,可以在$O(|X| \times w)$的时间内
计算出所有的子序列对称度。然而,如果想判别出真正的对称时间子序列,
还需要至关重要的一步——确定对称度阈值。
图~\ref{fig:lontitude_symmetry}展示了运输车经度标准化处理后的时间序列和对应的分段对称度变化情况。
一般工业场景中,对称度阈值往往是由领域专家输入的。但是,并非所有的
应用场景中都能找到专业的领域专家。并且,如果对称度阈值提供的不合适,
将极大的影响对称模式的挖掘。因此,本节提出了一个基于时间序列数据特征
和对称度分布特征的对称度阈值计算方法。
\begin{figure}
  \centering
  \includegraphics[width=0.86\linewidth]{std_lo_symmetry.png}
  \caption{标准化经度分段对称度变化}
  \label{fig:lontitude_symmetry}
\end{figure}

分段对称度阈值是对所有的时间子序列度量对称性,
因此,除了全局对称度阈值
考虑到的时间序列数据特征,对称度自然分布同样产生阈值划分。
基于时间序列数据特征的阈值确定算法已在3.1.2节中详述,本节只
讨论基于分布特征的对称度阈值确定算法。

在同一类时间序列中,长度相同的对称子序列的对称度较小,
而非对称子序列的对称度较大,这之间存在一个天然的阈值划分。
根据聚类方法[10]和自然断点分类的划分原则[11,12],
同一个类簇中数据的相似度高而不同类簇中数据的相似度低。
换言之,两个对称子序列的对称度差距应尽可能小,
而对称子序列与非对称子序列的对称度差距应尽可能大。
在概率统计中,方差用于衡量数据集中数据的偏离程度。
因此,很多算法使用方差作为度量指标变化的阈值[9]。
公式~\ref{eq:threshold2}展示了基于方差度量的对称度阈值确定原则,
其中,$DP_i$和$DP_j$均表示计算得到的子序列对称度,$\overline{DP_{i}}$̅和$\overline{DP_{j}}$
表示二者的均值。公式的最优化目标为在子序列对称度组成的集合中,
通过选择某个合适的值作为对称阈值,对称度小于该阈值的子序列为
对称子序列集合,对称度大于该阈值的子序列为非对称子序列集合,
使得对称子序列集合和非对称子序列集合的对称度方差之和最小。
图~\ref{fig:natural_break}展示了在运输车子序列对称度和挖掘机工况子序列对称度中
选择合适的自然断点,可使得对称模式和非对称模式的均值适中,
方差之和最小。综合时间序列数据特征和对称度分布的两类对称度阈值,
最终可以得到对称度阈值的完整公式,即式~\ref{eq:threshold}所示。
\begin{equation}
  \theta_{2}=\underset{x \in D P}{\operatorname{argmin}}\left(\sum_{D P_{i} \leq x}\left(D P_{i}-\overline{D P_{l}}\right)^{2}+\sum_{D P_{j}>x}\left(D P_{j}-\overline{D P_{j}}\right)^{2}\right)
  \label{eq:threshold2}
\end{equation}
\begin{equation}
  \theta=\min \left(\theta_{1}, \theta_{2}\right)
  \label{eq:threshold}
\end{equation}
\begin{figure}
  \centering
  \subcaptionbox{运煤车轨迹子序列对称度\label{fig:natural_break-a}}
  {\includegraphics[width=0.43\linewidth]{natural_break-a.png}}
  \subcaptionbox{挖掘机工况子序列对称度\label{fig:natural_break-b}}
  {\includegraphics[width=0.43\linewidth]{natural_break-b.png}}
  \caption{运输车轨迹和挖掘机工况子序列对称度分布}
  \label{fig:natural_break}
\end{figure}

算法~\ref{alg:threshold}给出了对称度阈值的算法,考虑到
基于对称度分布的阈值确定算法需要用到
动态集合方差的计算,本文采用了流式的方差计算方法。
第1行按照大小顺序对对称度进行排序,为方差的流式计算做准备。
第2-8行根据流式算法计算前i小的对称度方差并保存到数组$l$中,从而得到
顺序排序的前缀方差数组。
第9-16行将对称度倒序排序后计算前$i$大的对称度方差并保存到数组$r$中,
由此得到倒序排序的前缀方差数组。
第17-28行通过计算前$i$小和后$n-i$大的对称度方差之和的最小值
得到基于对称度分布的阈值。
第24-35行计算时间序列相邻点距离的均值得到全局对称度阈值,
并通过和对称度分布阈值比较得到较小者,作为最终的对称度阈值。
采用这种算法得到的对称度阈值不仅考虑到了时间序列本身的特征,
还考虑到了对称度的分布,在实验结果中有良好的表现。


\renewcommand{\algorithmicrequire}{\textbf{输入:}\unskip}
\renewcommand{\algorithmicensure}{\textbf{输出:}\unskip}

\begin{algorithm}[t]
  \caption{对称度阈值划分算法$calculate\_threshold$}
  \label{alg:threshold}
  \small
  \begin{algorithmic}
    \REQUIRE 子序列对称度列表$y$,时间序列$X=\left(p_{1}, p_{2}, \dots, p_{n}\right)$
    \ENSURE 对称度阈值$\theta$

    \STATE sort$(y)$
    \STATE $a_1 \leftarrow y_1$
    \STATE $i \leftarrow 2$
    \WHILE{$i \leq \left|y\right|$}
    \STATE $a_i \leftarrow a_{i-1}+(y_i-a_{i-1})/{i}$
    \STATE $l_i \leftarrow (i-1) / (i \times i) \times(y_i-a_{i-1})^{2}+(i-1) / i \times l_{i-1}$
    \STATE $i \leftarrow i+1$
    \ENDWHILE

    \STATE reverse$(y)$
    \STATE $a_1 \leftarrow y_1$
    \STATE $i \leftarrow 2$
    \WHILE{$i \leq \left|y\right|$}
    \STATE $a_i \leftarrow a_{i-1}+(y_i-a_{i-1})/{i}$
    \STATE $r_i \leftarrow (i-1) / (i \times i) \times(y_i-a_{i-1})^{2}+(i-1) / i \times r_{i-1}$
    \STATE $i \leftarrow i+1$
    \ENDWHILE

    \STATE reverse$(r)$
    \STATE reverse$(y)$
    \STATE $d \leftarrow l_1 + r_2$
    \STATE $idx \leftarrow 1$
    \STATE $i \leftarrow 2$
    \WHILE{$i < \left|X\right|$}
    \IF{$l_i + r_{i+1} < d$}
    \STATE $d \leftarrow l_i + r_{i+1}$
    \STATE $idx \leftarrow i$
    \ENDIF
    \STATE $i \leftarrow i+1$
    \ENDWHILE
    \STATE $i \leftarrow 1$
    \WHILE{$i < \left|X\right|$}
    \STATE minus $\leftarrow$ minus $+D\left(p_{i}, p_{i+1}\right)$
    \STATE $i \leftarrow i+1$
    \ENDWHILE
    \STATE $\theta \leftarrow \min \left({\text { minus }} / (n-1), {dp_{idx}} / {n}\right)$
    \RETURN $\theta$
  \end{algorithmic}
\end{algorithm}

\subsection{挖掘分段对称模式}
计算出时间序列$X$所有在长度约束范围之内的子序列对称度之后,
可以利用贪心算法挖掘得到所有满足对称性的子序列。以斗杆外摆为例,
在挖掘过程中,存在对称子序列相互包含的情况。如图~\ref{fig:overlap}
中所示,子序列$a$、$b$、$c$均是对称子序列。由于要挖掘不重叠对称子序列,
若结果集合中选择了$a$,则不能再选择$b$和$c$。因此,
为了充分利用时间序列的信息,本文以对称子序列的数量最大值为
优化目标。根据贪心算法的思想,若上一个选择的对称子序列为$s_i$
其长度为$w$,则下一个对称子序列的贪心策略为从数据点$i+w$
开始,选择第1个长度为$w$的对称子序列, 可以保证能得到数量最多的
不重叠子序列。换言之,对于两个长度为$w$且存在重叠的时间序列,选择开始点最早的
时间序列能保证挖掘到数量最多的对称子序列。具体证明如下:假设存在3个长度为$w$的对称子序列
$A=\left(p_{i}, p_{i+1}, \dots, p_{i+w-1}\right)$,$B=\left(p_{j}, p_{j+1}, \dots, p_{j+w-1}\right)$和
$C=\left(p_{k}, p_{k+1}, \dots, p_{k+w-1}\right)$,并且$i<j<k$。若选择A序列进入
对称模式集合,则剩余对称模式可以从$p_{i+w}$之后的点进行选取。若选择B序列进入
对称模式集合,则剩余对称模式可以从$p_{j+w}$之后的点进行选取。由$i<j$可知,
$i+w<j+w$,前者的选择范围比后者更广,如果在选择B序列进入对称模式集合之后仍可选择C序列,
证明$j+w \leq k$,则$i+w$也$\leq k$,同样也可以选择A序列来替换B序列进入对称模式集合。
反之,如果选择A序列进入对称模式集合,则不一定可以由B序列替换。因此,对于长度相同的对称子序列,
选择起始时间点最早的序列能保证挖掘出最多的对称模式。

\begin{figure}
  \centering
  \includegraphics[width=0.86\linewidth]{symmetric_overlap.png}
  \caption{对称时间序列反映重叠现象的案例}
  \label{fig:overlap}
\end{figure}

本文根据上述对称模式挖掘方法提出了相应的具体算法,
如算法3.2所示。第1-19行根据第3.2节提出的对称度计算方式
计算了给定时间长度约束下所有子序列的对称度,第20行根据算法~\ref{alg:threshold}的计算方式
确定了对称度阈值,第21-30行根据贪心策略
计算出数量最多的不重叠对称子序列。由前述定义可知,第1-19行计算子序列对称度的时间复杂度为$O(|X| \times w)$,
第20行自然断点的确定和不重叠对称子序列的挖掘均只需要$O(|X|)$的时间。
因此,本算法的整体时间复杂度为$O(|X| \times w)$,其效率远远高于直接利用
DTW 计算的$O\left(|X| \times w^{2}\right)$,为后续的数据分析提供数据基础。
\renewcommand{\algorithmicrequire}{\textbf{输入:}\unskip}
\renewcommand{\algorithmicensure}{\textbf{输出:}\unskip}

\begin{algorithm}[t]
  \caption{时间序列对称模式挖掘算法$calculate\_symmtric\_pattern$}
  \label{alg:symmetric_pattern}
  \small
  \begin{algorithmic}
    \REQUIRE 时间序列$X=\left(p_{1}, p_{2}, \dots, p_{n}\right)$,长度约束$w$
    \ENSURE 对称模式$Q$

    \STATE $i \leftarrow 1$
    \WHILE{$i \leq \left|X\right|$}
    \STATE $dp_{i,i} \leftarrow inf$
    \ENDWHILE

    \STATE $i \leftarrow 1$
    \WHILE{$i < \left|X\right|$}
    \STATE $dp_{i,i+1} \leftarrow D\left(p_{i}, p_{i+1}\right)$
    \ENDWHILE

    \STATE $len \leftarrow 3$
    \WHILE{$len \leq \left|X\right|$}
    \STATE $i \leftarrow len$
    \WHILE{$i \leq \left|X\right|-len+1$}
    \STATE $dp_{i,i+len-1} = D\left(p_{i}, p_{i+1}\right)+\min \left(dp_{i,i+len-2},dp_{i+1,i+len-1},dp_{i+1,i+len-2}\right)$
    \ENDWHILE
    \ENDWHILE

    \STATE $i \leftarrow 1$
    \WHILE{$i \leq \left|X\right|-w+1$}
    \STATE $y_i=dp_{i,i+w-1}$
    \ENDWHILE
    \STATE $\theta = calculate\_threshold\left(y,X\right)$

    \STATE $i \leftarrow 1$
    \WHILE{$i \leq \left|X\right|-w+1$}
    \IF{$dp_{i,i+w-1}/w \leq \theta$}
    \STATE 把$S=\left(p_{i+1}, p_{i+2}, \dots, p_{i+w-1}\right)$放入对称子序列集合$Q$
    \STATE $i \leftarrow i+w$
    \ELSE
    \STATE $i \leftarrow i+1$
    \ENDIF
    \ENDWHILE
    \RETURN $Q$
  \end{algorithmic}
\end{algorithm}

\section{本章小结}
本章主要介绍了时间序列对称模式挖掘的整体框架和算法细节。
首先定义了对于完整的全局时间序列如何度量对称性,并通过时间序列
数据特征确定了对称度阈值,由阈值判断时间序列是否属于全局对称模式。
然后结合滑动窗口将全局时间序列对称性度量算法推广到了分段时间序列中,
接下来从时间序列数据特征和对称度分布特征确定了分段时间序列对称度阈值,
并根据对称度阈值过滤得到所有的对称子序列。最后利用
贪心策略过滤具有重叠部分的对称子序列,从而得到对称模式。
图~\ref{fig:algorithm_process}展示了对称模式挖掘的算法流程,
在对时间序列数据进行标准化之后,通过选择是否度量时间序列
全局对称性,采用不同的对称模式挖掘算法。
根据第5章的实验证明,本节所定义的全局和分段对称模式挖掘算法
在不同来源的数据集上具有更高的准确性和鲁棒性。
而且,由于采用了区间动态规划的思想进行优化,
分段对称模式挖掘算法在时间上比基于动态时间规整的算法高出了一个阶数,
在实际工业场景具有广泛的应用价值。
\begin{figure}
  \centering
  \includegraphics[width=0.86\linewidth]{algorithm_process.png}
  \caption{时间序列对称模式挖掘流程图}
  \label{fig:algorithm_process}
\end{figure}