% !TeX root = ../thuthesis-example.tex

% 中英文摘要和关键字

\begin{abstract}
  信息技术和工业经济的融合催生出了智能化和数字化的现代工业,
  同时也产生了规模庞大的工业数据。
  时间序列是一种典型的工业大数据,蕴含着丰富的
  规律和模式。其中,对称模式作为一种
  具有前后对应特征的时序模式,广泛分布于多种时间序列之中。
  挖掘对称模式在轨迹分析、异常识别和序列预测等领域具有重要的
  研究价值。然而,工业时间序列的随机误差和超大数据量给对称模式
  的挖掘带来了困难。基于此,
  本文研究了一类时间和空间效率均较为高效的
  对称模式挖掘方法来应对上述问题。

  % 再度量子序列的对称性
  % 对称模式涉及到时间序列的对比与匹配。
  % 因此,时间序列的相似性度量对
  % 对称模式挖掘算法的效果和性能具有决定性作用。
  % 一种简单的序列相似性度量方法是欧式距离。
  % 但是欧氏距离采用一一对应的点匹配策略,
  % 对时间序列偏移、缺失和形状变化等比较敏感。
  % 动态时间扭曲(DTW)通过弯曲时间序列使得每个点有多个匹配候选,
  % 从而提高了相似性度量方法的鲁棒性。
  % 但工业应用中时间序列的数据量往往高达几十甚至上百GB,
  % 动态时间扭曲方法的计算效率堪忧。

  对称模式分为全局对称模式和分段对称模式。
  对于全局对称模式而言,准确的对称性度量算法是其关节所在。
  欧氏距离度量是一种简单而有效的方法,
  但其一一对应的点匹配策略,对时间序列
  偏移、缺失和形状变化比较敏感。
  本文提出了一个基于最优匹配和自适应阈值的
  全局对称模式挖掘算法,对时间序列各类异常具有很好的鲁棒性。
  经过实验证明,本方法在
  对称模式挖掘效果上超过了
  利用欧氏距离和动态时间扭曲度量对称性的方法。并且,
  本文还通过优化对称度计算方式将全局对称模式挖掘算法扩展到了
  流式时间序列,在保证时间性能的同时,将算法的空间复杂度
  优化至了$O(n)$,为对称模式挖掘在大数据上的应用提供了可能。

  分段对称模式挖掘除了要度量对称性并确定阈值,
  还需要对时间序列进行分段处理并解决子模式重叠问题。
  因此,根据区间动态规划算法思想,
  结合滑动窗口模型和贪心策略,本文设计了一种渐进时间复杂度为
  $O\left( \left| X \right| \times w \right)$的
  分段对称模式挖掘算法。
  经过实验证明,
  本文提出的分段算法不但保持了模式挖掘效果的准确性,
  而且相比基于动态时间扭曲的算法在效率上
  有了100倍以上的提升。
  此外,根据IoTDB的存储和计算方式,本文实现了基于
  UDF和元数据的两种分段算法,分别适用于写入和查询负载较重
  的应用场景,在系统层面对分段对称模式挖掘算法进行了扩展。

  通过对全局和分段对称模式挖掘算法的研究与实验,我们发现,
  本文提出的方法不仅在挖掘效果上具有很好的准确性和鲁棒性,
  在性能上也高于基于动态时间扭曲的算法,
  而且在复杂应用场景有很好的可扩展性。
  最后,本文在IoTDB系统上实现了对称模式挖掘算法,
  完善了IoTDB挖掘特定时间序列模式的功能,
  为用户数据分析提供有价值的信息。
  
  % 关键词用“英文逗号”分隔,输出时会自动处理为正确的分隔符
  \thusetup{
    keywords = {时间序列, 相似性度量, 对称模式, 动态规划},
  }
\end{abstract}

\begin{abstract*}
  The integration of information technology and industrial 
  economy has given birth to intelligent and digitized 
  modern industries, as well as large-scale industrial 
  data. Time series is a typical industrial big data, 
  which contains rich laws and patterns. Among them, 
  the symmetric pattern is widely distributed in various 
  time series as a time series pattern with front and back 
  correspondence characteristics. Mining symmetric patterns 
  has important research value in the fields of trajectory 
  analysis, anomaly identification and sequence prediction. 
  However, the random errors and large data volumes of 
  industrial time series bring difficulties to the mining 
  of symmetric patterns. Based on this, this paper studies 
  a class of time- and space-efficient symmetric pattern 
  mining methods to deal with the above problems.

  Symmetric patterns are divided into global symmetric 
  patterns and segmented symmetric patterns. For global 
  symmetric patterns, an accurate symmetry measure algorithm 
  is where the joints lie. The Euclidean distance metric 
  is a simple and effective method, but its one-to-one 
  correspondence point matching strategy is sensitive to 
  time series offset, missing and shape changes. This paper 
  proposes a global symmetric pattern mining algorithm based 
  on optimal matching and adaptive threshold, which has good 
  robustness to various anomalies in time series. 
  Experiments show that this method surpasses the method of 
  using Euclidean distance and dynamic time warp to measure 
  symmetry in the effect of symmetric pattern mining. In 
  addition, this paper also extends the global symmetric 
  pattern mining algorithm to streaming time series by 
  optimizing the calculation method of symmetry degree. 
  While ensuring the time performance, the space complexity 
  of the algorithm is optimized to $O(n)$, which provides 
  the possibility for the application of symmetric pattern 
  mining in big data.

  In addition to measuring symmetry and determining 
  thresholds, segmented symmetric pattern mining 
  also needs to segment the time series and solve 
  the subpattern overlap problem. Therefore, 
  according to the idea of interval dynamic programming 
  algorithm, combined with sliding window model and greedy 
  strategy, this paper designs a segmented symmetric pattern
  mining algorithm with asymptotic time 
  complexity $O\left( \left| X \right| \times w \right)$.
  Experiments show that the segmented algorithm proposed 
  in this paper not only maintains the accuracy of the 
  pattern mining effect, but also improves the efficiency 
  by more than 100 times compared with the algorithm 
  based on dynamic time warping. In addition, 
  according to the storage and calculation methods of IoTDB, 
  this paper implements two segmentated algorithms 
  based on UDF and metadata, which are suitable for 
  application scenarios with heavy write and 
  query loads, which extends segmented symmetric 
  pattern mining algorithm in system level.

  Through research and experiments on global and 
  segmented symmetric pattern mining algorithms, 
  we found that the method proposed in this paper 
  not only has good accuracy and robustness in mining 
  effect, but also has higher performance than the 
  algorithm based on dynamic time warping, 
  and has good scalability in complex application scenarios. 
  Finally, this paper implements a symmetric pattern mining 
  algorithm on the IoTDB system, improves the function 
  of IoTDB to mine specific time series patterns, 
  and provides valuable information for user data analysis.
  % Use comma as separator when inputting
  \thusetup{
    keywords* = {time series, symmetric pattern, similarity measure, dynamic programming},
  }
\end{abstract*}
