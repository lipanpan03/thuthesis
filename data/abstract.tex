% !TeX root = ../thuthesis-example.tex

% 中英文摘要和关键字

\begin{abstract}
  随着信息化和工业化的融合,物联网和工业互联网蓬勃发展,
  由此产生了以时间序列为代表的大量工业大数据。
  时间序列中蕴含着很多有价值的模式,对称模式作为一种
  具有前后对应特征的时序模式,在各类时间序列中广泛存在。
  挖掘对称模式在行为分析、轨迹跟踪、异常检测等领域具有重要的
  研究价值。考虑到大规模的时间序列数据给计算资源和运行效率
  带来的挑战,本文将研究一类时间和空间效率均较为高效的
  对称模式挖掘方法来应对上述问题。

  % 再度量子序列的对称性
  % 对称模式涉及到时间序列的对比与匹配。
  % 因此,时间序列的相似性度量对
  % 对称模式挖掘算法的效果和性能具有决定性作用。
  % 一种简单的序列相似性度量方法是欧式距离。
  % 但是欧氏距离采用一一对应的点匹配策略,
  % 对时间序列偏移、缺失和形状变化等比较敏感。
  % 动态时间扭曲(DTW)通过弯曲时间序列使得每个点有多个匹配候选,
  % 从而提高了相似性度量方法的鲁棒性。
  % 但工业应用中时间序列的数据量往往高达几十甚至上百GB,
  % 动态时间扭曲方法的计算效率堪忧。

  对称模式分为全局对称模式和分段对称模式。
  对于全局对称模式而言,准确的对称性度量算法是其关节所在。
  欧氏距离度量是一种简单而有效的方法,
  但是其一一对应的点匹配策略,对时间序列偏移、缺失和形状变化
  比较敏感。本文提出了一个基于全局最优匹配的
  时间序列对称性度量算法来计算时间序列整体的对称性,
  再根据时间序列的数据特征确定对称度阈值以分类得到
  具有全局对称性的时间序列。通过在不同来源的多种数据集上
  进行实验证明,本方法在
  对称模式挖掘效果上超过了
  利用欧氏距离和动态时间扭曲度量对称性的方法。

  分段对称模式挖掘除了要计算序列对称性并确定阈值,
  还需要对时间序列进行分段处理并最后解决子模式重叠问题。
  因此,基于区间动态规划算法思想的启发,本文提出了一种能够
  在$O\left( \left| X \right| \times w \right)$的
  时间复杂度内挖掘出时间序列分段对称模式的方法。
  具体来说,给定对称模式长度约束,
  以长度单调递增的顺序计算出子序列对称度,
  进而依据贪心策略选择数量最多且不重叠对称子序列组成的分段对称模式。
  通过在1个合成数据集和3个真实数据集上的实验证明,
  本文提出的分段对称模式挖掘方法不但在模式挖掘效果上优于
  欧式距离度量方法,而且相比动态时间扭曲方法在效率上
  有了100倍以上的提升。

  本文还研究了对称模式挖掘算法在流式时间序列和
  变长对称模式场景下的扩展。通过对称性度量算法状态
  的流式转移和以空间换时间的策略,将单个时间序列的
  计算复杂度由$O\left( w^2 \right)$
  降为了$O\left( w \right)$。此外,本文还
  利用数据点差分的分布特征,
  设计了能自动调节大小的对称模式窗口以提高对称模式挖掘的完整性。
  最后,本文把对称模式挖掘算法进行了系统实现并集成到
  IoTDB-Quality中,完善了工具的功能并加快了
  工具挖掘对称模式的效率。
  
  % 关键词用“英文逗号”分隔,输出时会自动处理为正确的分隔符
  \thusetup{
    keywords = {时间序列, 对称模式, 相似性度量, 动态规划},
  }
\end{abstract}

\begin{abstract*}
  With the integration of informatization and industrialization, 
  the Internet of Things and the Industrial Internet have 
  flourished, resulting in a large amount of industrial big data 
  represented by time series. There are many valuable patterns 
  in time series. Symmetric patterns, as a series of time points 
  with corresponding features before and after, widely exist 
  in all kinds of time series. Mining symmetric patterns has 
  important research value for behavior analysis, trajectory 
  tracking, anomaly detection and other fields. Considering 
  the challenges brought by large-scale time series data to 
  computing resources and operational efficiency, this paper 
  will study a time- and space-efficient symmetric pattern 
  mining method to deal with the above problems.

  Symmetric patterns involve the comparison and matching of 
  segmented time series. Therefore, the similarity measure 
  of temporal subsequences plays a decisive role in the 
  temporal performance of the symmetric pattern mining algorithm. 
  A simple and effective measure of sequence similarity is 
  Euclidean distance. However, the amount of time series data 
  in industrial applications is often as high as tens or even 
  hundreds of GB, and the computational efficiency of the 
  similarity retrieval method based on Euclidean distance is 
  still worrying. Therefore, based on the inspiration of 
  the dynamic time warping (DTW) algorithm, this paper 
  proposes a method that can mine symmetric patterns of 
  time series within the time complexity of 
  $O\left( \left| X \right| \times w \right)$. 
  Specifically, given the length constraint as the size of 
  the symmetric pattern window, the symmetric subsequence 
  is calculated based on the interval dynamic programming 
  algorithm, and then the symmetric pattern composed of 
  the largest number of non-overlapping symmetric subsequences 
  is selected according to the greedy strategy. Experiments 
  show that the symmetric pattern mining method proposed 
  in this paper is not only better than the mining method 
  based on Euclidean distance in pattern mining effect, 
  but also has more than 100 times improvement in efficiency 
  compared with the mining method based on DTW distance.

  This paper also investigates the extension of the symmetric 
  pattern mining algorithm to streaming time series and 
  variable-length symmetric pattern scenarios. Streaming time 
  series can be regarded as a dynamic data collection that 
  grows infinitely with time. The symmetry of streaming time 
  series needs to be real-time. This paper uses the 
  state-optimized streaming algorithm calculated by 
  the improved DTW algorithm to reduce the computational 
  complexity of symmetry at each time point from 
  $O\left( w^2 \right)$ to $O\left( w \right)$. 
  In addition, there may be symmetric patterns of different 
  lengths in the same time series. Therefore, this paper uses 
  the distribution characteristics of data point differences 
  to design a symmetric pattern window that can automatically 
  adjust the size. Experiments show that the symmetric pattern 
  mining method with an adaptive window is more complete than 
  the original method and other methods.

  By studying symmetric pattern mining of time series, 
  we found that the symmetric pattern mining method based on 
  the improved DTW algorithm is not only better than the mining 
  method based on Euclidean distance in pattern mining effect, 
  but also significantly better than the mining method based 
  on DTW in performance. Finally, this paper systematically 
  implements the proposed algorithm and integrates it into 
  IoTDB-Quality, which improves the function of the tool and 
  accelerates the efficiency of the tool to mine symmetric 
  patterns.
  % Use comma as separator when inputting
  \thusetup{
    keywords* = {time series, symmetric pattern, similarity measure, dynamic programming},
  }
\end{abstract*}
