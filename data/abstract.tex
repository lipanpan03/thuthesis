% !TeX root = ../thuthesis-example.tex

% 中英文摘要和关键字

\begin{abstract}
  信息技术和工业经济的融合催生出了智能化和数字化的现代工业,
  同时也产生了规模庞大的工业数据。
  时间序列是一种典型的工业大数据,蕴含着丰富的
  规律和知识。其中,对称子段作为一种
  具有前后对应特征的时序子段,广泛分布于多种时间序列之中。
  挖掘对称子段在轨迹分析、异常识别和序列预测等领域具有重要的
  研究价值。然而,工业时间序列的随机误差和超大数据量给
  对称子段的挖掘带来了困难。基于此,
  本文研究了一类时间和空间效率均较为高效的
  对称子段挖掘算法来应对上述问题。

  % 再度量子序列的对称性
  % 对称模式涉及到时间序列的对比与匹配。
  % 因此,时间序列的相似性度量对
  % 对称模式挖掘算法的效果和性能具有决定性作用。
  % 一种简单的序列相似性度量算法是欧式距离。
  % 但是欧氏距离采用一一对应的点匹配策略,
  % 对时间序列偏移、缺失和形状变化等比较敏感。
  % 动态时间扭曲(DTW)通过弯曲时间序列使得每个点有多个匹配候选,
  % 从而提高了相似性度量算法的鲁棒性。
  % 但工业应用中时间序列的数据量往往高达几十甚至上百GB,
  % 动态时间扭曲算法的计算效率堪忧。

  对称子段判定算法是解决对称子段挖掘问题的基础。
  对于对称子段判定问题,准确的对称性度量算法是其关节所在。
  欧氏距离度量是一种简单而有效的算法,
  但其一一对应的点匹配策略,对时间序列
  偏移、缺失和形状变化比较敏感。
  本文提出了一个基于最优匹配和自适应阈值的
  对称子段判定算法,对时间序列各类异常具有很好的鲁棒性。
  经过实验证明,本算法在
  对称子段判定效果上超过了
  利用欧氏距离和动态时间扭曲度量对称性的算法。并且,
  本文还通过优化对称度计算方式将对称子段判定算法扩展到了
  流式时间序列,在保证时间性能的同时,将算法的空间复杂度
  优化至了$O(n)$,为对称子段判定在大数据上的应用提供了可能。

  对称子段挖掘除了要对子段对称性进行判定,
  还需要对时间序列进行分段处理并解决子段重叠问题。
  因此,根据区间动态规划算法思想,
  结合滑动窗口模型和贪心策略,本文设计了一种渐进时间复杂度为
  $O\left( \left| X \right| \times w \right)$的
  对称子段挖掘算法。
  经过实验证明,
  本文提出的算法不但保持了对称子段挖掘效果的准确性,
  而且相比基于动态时间扭曲的算法在效率上
  有了100倍以上的提升。
  此外,根据IoTDB的存储和计算方式,本文设计了基于
  UDF和元数据的两种算法实现,分别适用于写入和查询负载较重
  的应用场景,在系统层面对对称子段挖掘算法进行了扩展。

  通过对对称子段判定和挖掘算法的研究与实验,我们发现,
  本文提出的算法不仅在挖掘效果上具有很好的准确性和鲁棒性,
  在性能上也高于基于动态时间扭曲的算法,
  而且在复杂应用场景有很好的可扩展性。
  最后,本文在IoTDB系统上实现了对称子段判定和挖掘算法,
  完善了IoTDB挖掘时间序列特定类型子段的功能,
  为用户数据分析提供有价值的信息。
  
  % 关键词用“英文逗号”分隔,输出时会自动处理为正确的分隔符
  \thusetup{
    keywords = {时间序列, 相似性度量, 对称子段, 动态规划},
  }
\end{abstract}

\begin{abstract*}
  The integration of information technology and 
  industrial economy has given birth to intelligent 
  and digital modern industries, as well as 
  large-scale industrial data. Time series is a 
  typical industrial big data, which contains rich 
  laws and knowledge. Among them, the symmetric 
  subsegment, as a type of time series subsegment 
  with corresponding features before and after, 
  is widely distributed in various time series. 
  Mining symmetric subsegments has important research 
  value in the fields of trajectory analysis, 
  anomaly identification and sequence prediction. 
  However, the random errors and large data volume of 
  industrial time series bring difficulties to the mining 
  of symmetric subsegments. Based on this, this paper 
  studies a type of symmetric subsegment mining 
  methods that are both efficient in time and space 
  to deal with the above problems.

  Symmetric subsegment judgment algorithm is the 
  basis to solve the problem of symmetric subsegment 
  mining. For the symmetric subsegment judgment problem, 
  the accurate symmetry measurement algorithm is 
  where its joints are located. Euclidean distance 
  metric is a simple and effective method, but its 
  one-to-one correspondence point matching strategy 
  is sensitive to time series offset, missing and 
  shape changes. In this paper, a symmetric subsegment 
  judgment algorithm based on optimal matching and 
  adaptive threshold is proposed, which has good 
  robustness to various anomalies in time series. 
  Experiments show that this method surpasses the 
  method of using Euclidean distance and dynamic 
  time warping to measure symmetry in judging the 
  effect of symmetric subsegemnts. In addition, 
  this paper also extends the symmetric subsegment 
  judgment algorithm to streaming time series 
  by optimizing the calculation method of symmetry 
  degree. While ensuring the time performance, 
  the space complexity of the algorithm is optimized 
  to $O(n)$, for the symmetric subsegment judgment 
  in the application of big data provides the 
  possibility.

  In addition to judging the symmetry of subsegments, 
  symmetric subsegment mining also needs to segment 
  the time series and solve the problem of overlapping 
  subsegments. Therefore, according to the idea of 
  interval dynamic programming algorithm, combined 
  with sliding window model and greedy strategy, 
  this paper designs a symmetric subsegment mining 
  algorithm with asymptotic time complexity 
  $O\left( \left| X \right| \times w \right)$. 
  Experiments show that the mining algorithm proposed 
  in this paper not only maintains the accuracy of 
  the subsegment mining methods, but also improves 
  the efficiency by more than 100 times compared 
  with the algorithm based on dynamic time warping. 
  In addition, according to the storage and computing 
  methods of IoTDB, this paper implements two mining 
  algorithms based on UDF and metadata, which are 
  suitable for application scenarios with heavy write 
  and query loads respectively, and the symmetric 
  subsegment mining algorithm is extended at the 
  system level.

  Through the research and experiments on the 
  symmetric subsegment judgment and mining algorithm, 
  we found that the method proposed in this paper 
  not only has good accuracy and robustness in 
  mining effect, but also has higher performance 
  than the algorithm based on dynamic time warping. 
  And it has good scalability in complex application 
  scenarios. Finally, this paper implements the 
  symmetric subsegment judgment and mining algorithm 
  on the IoTDB system, improves the function of IoTDB 
  to mine specific time series subsegments, and 
  provides valuable information for user data analysis.

  % Use comma as separator when inputting
  \thusetup{
    keywords* = {time series, symmetric subsegment, similarity measure, dynamic programming},
  }
\end{abstract*}
